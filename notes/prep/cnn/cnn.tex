\documentclass[12pt]{article}
\usepackage{fullpage,enumitem,amsmath,amssymb,graphicx}

\newcommand{\ub}{\boldsymbol{u}}
\newcommand{\vb}{\boldsymbol{v}}

\begin{document}

\begin{center}
{\Large CNNs - Stanford CS231N}

\begin{tabular}{rl}
Name: & Eli Andrew
\end{tabular}
\end{center}

  \begin{itemize}
    \item \textbf{Gradient Formulas}
    \begin{itemize}
        \item $\frac{\partial L}{\partial x} = (\frac{\partial L}{\partial y})w^T$
        \item $\frac{\partial L}{\partial w} = x^T(\frac{\partial L}{\partial y})$
    \end{itemize}
    \item \textbf{Spatial Arrangement}
    \begin{itemize}
        \item \textbf{Depth}
        \begin{itemize}
            \item Depth of output volume is a hyperparameter
            \item It corresponds to the number of filters we use (each one looking for something different)
            \item \textbf{Depth Column:} set of neurons looking at the same region of input
        \end{itemize}
        \item \textbf{Stride}
        \begin{itemize}
            \item Stride is how many pixels you move each filter after each convolution with the input
            \item Larger stride will produce smaller output volumes
        \end{itemize}
        \item \textbf{Zero Padding}
        \begin{itemize}
            \item Hyperparameter for padding input volume with zeros
            \item Allows for controlling the size of the output volume
        \end{itemize}
        \item Output volume of a convolution is given as:
        \begin{gather*}
            \frac{W - F + 2P}{S} + 1
        \end{gather*}
        \item Where $W$ is the input volume size, $F$ is the filter size, $P$ is zero padding size, and $S$ is stride
        \item Example with input size $7\times7$ filter size $3\times3$, stride 1 and pad 0 giving an output size of $5\times5\times3$:
        \begin{gather*}
            W = 7, F = 3, S = 1, P = 0 \\ \\
            \frac{W - F + 2P}{S} + 1 = \frac{7 - 3 + 0}{1} + 1 = 5
        \end{gather*}
        \item Setting zero-padding to: $P = \frac{F-1}{2}$ when $S = 1$ keeps input and output as the same dimension
    \end{itemize}
  \end{itemize}


\end{document}
